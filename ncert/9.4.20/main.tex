\let\negmedspace\undefined
\let\negthickspace\undefined
\documentclass[journal]{IEEEtran}
\usepackage[a5paper, margin=10mm, onecolumn]{geometry}
%\usepackage{lmodern} % Ensure lmodern is loaded for pdflatex
\usepackage{tfrupee} % Include tfrupee package

\setlength{\headheight}{1cm} % Set the height of the header box
\setlength{\headsep}{0mm}     % Set the distance between the header box and the top of the text

\usepackage{gvv-book}
\usepackage{gvv}
\usepackage{cite}
\usepackage{amsmath,amssymb,amsfonts,amsthm}
\usepackage{algorithmic}
\usepackage{graphicx}
\usepackage{textcomp}
\usepackage{xcolor}
\usepackage{txfonts}
\usepackage{listings}
\usepackage{enumitem}
\usepackage{mathtools}
\usepackage{gensymb}
\usepackage{comment}
\usepackage[breaklinks=true]{hyperref}
\usepackage{tkz-euclide} 
\usepackage{listings}
% \usepackage{gvv}                                        
\def\inputGnumericTable{}                                 
\usepackage[latin1]{inputenc}                                
\usepackage{color}                                            
\usepackage{array}                                            
\usepackage{longtable}                                       
\usepackage{calc}                                             
\usepackage{multirow}                                         
\usepackage{hhline}                                           
\usepackage{ifthen}                                           
\usepackage{lscape}

\renewcommand{\thefigure}{\theenumi}
\renewcommand{\thetable}{\theenumi}
\setlength{\intextsep}{10pt} % Space between text and floats


\numberwithin{equation}{enumi}
\numberwithin{figure}{enumi}
\renewcommand{\thetable}{\theenumi}

% Marks the beginning of the document
\begin{document}
\bibliographystyle{IEEEtran}

\title{9.4.20}
\author{EE24BTECH11049 - Patnam Shariq Faraz Muhammed}

% \maketitle
% \newpage
% \bigskip
{\let\newpage\relax\maketitle}

\textbf{Question}:\\

In a bank, the principal continuously increases at a rate of $r\%$ per year. Find the value of $r$ if Rs 100 doubles in 10 years \brak{\log_e2 = 0.6931}.\\

\textbf{Solution: }\\

\begin{table}[ht!]
    \centering
    \input{tables/table}
    \caption{Variables used}
    \label{tab:my_label}
\end{table}

$P$ is the principal at any time, according to the given question, the rate of change of principal can be written as follows.

\begin{align}
    \frac{dP}{dt} &= \brak{\frac{r}{100}} \times P \label{1}
\end{align}

Separation of the variables in the equation \eqref{1}

\begin{align}
     \frac{dP}{p} &= \brak{\frac{r}{100}} \times dt \label{2}
\end{align}

Integration on both sides \eqref{2}

\begin{align}
    \int{\frac{dP}{p}} &= \int{\brak{\frac{r}{100}} dt}\\
    \log_e{P} &= \frac{rt}{100} + C\\
    P &= e^{\frac{rt}{100} + C}\\
    P &= e^{\frac{rt}{100}}e^C\\
    P &= C_1e^{\frac{rt}{100}} \label{3}
\end{align}

At time $t = 0$, it is given that the principal is $100$, that is, $P_0 = 100$. \\ 
Substitute in equation \eqref{3}

\begin{align}
    100 = C_1
\end{align}

Principal can be written as

\begin{align}
    P &= 100 \times e^{\frac{rt}{100}} \label{4}
\end{align}

At $t = 10$, the principal doubles, that is, $P = 200$, using equation \eqref{4}

\begin{align}
    200 &= 100 \times e^{\frac{r}{10}}\\
    2 &= e^{\frac{r}{10}}\\
    \log_e{2} &= \frac{r}{10}\\
    r &= 10 \times \log_e{2}\\
    r &= 6.931
\end{align}

\textbf{Verification of the solution computationally:}

From the definition of $\frac{dy}{dx}$,
\begin{align}
    \frac{dy}{dx} \approx \frac{y\brak{x + h} - y\brak{x}}{h}
\end{align}

rearrange the terms 

\begin{align}
    y\brak{x + h} &= y\brak{x} + h \times \frac{dy}{dx}
\end{align}

comparing the terms

\begin{align}
    P\brak{t + h} &= P\brak{t} + h \times \frac{P}{10}
\end{align}

Let $\brak{t_0, P_0}$ be the initial conditions then 

\begin{align}
    t_1 &= t_0 + h\\
    P_1 &= P_0 + h \times \frac{P_0}{10}
\end{align}

Generalizing this 

\begin{align}
    t_{n+1} &= t_{n}+h \\
    P_{n+1} &= P_{n}+h \times \frac{P_n}{10}
\end{align}
Where h is step and is small.
We iterate this by taking the initial conditions from $t = 0$ to $t = 10$\\
By plotting all the points \brak{t,P} we get the graph of function P varying with t.\\
The comparison between theoretical and simulation curves is shown in the figure, we can clearly see that both the curves are coincides which verifies our solution.\\
For the following approximate graph, I chose $h = 0.1$

\begin{figure}[H]
   \centering
   \includegraphics[width=0.7\columnwidth]{figs/fig.png}
    \caption{Approximate solution of the DE}
\end{figure}
\end{document}
